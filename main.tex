\documentclass[11pt]{article}
\usepackage{amsmath}
\usepackage{hyperref}
\usepackage{lineno}
\usepackage[stretch=10,shrink=10]{microtype}
\usepackage{xcolor}
\usepackage[margin=1.2in]{geometry}
\usepackage{enumitem}

% colored inline code
\let\oldttfamily\ttfamily
\renewcommand{\ttfamily}{\oldttfamily\color[HTML]{9C471D}}

% smaller line numbers, lighter gray, further to the left
\renewcommand\linenumberfont{\normalfont\tiny\color{gray}}
\setlength\linenumbersep{2em}

\linenumbers

\newenvironment{answer}{
\begin{quote}}{
\end{quote}}

\usepackage{import}
\usepackage{xifthen}
\usepackage{pdfpages}
\usepackage{transparent}

\newcommand{\incfig}[1]{%
  \def\svgwidth{\columnwidth}
  \import{./figures/}{#1.pdf_tex}
}

% =====

\title{Exercises for ISLP\footnote{\textit{An Introduction to
    Statistical Learning with Applications in Python}:
\url{https://www.statlearning.com/}}}
\author{Kevin Ma}
\date{May 2025}

\begin{document}

\maketitle

\setcounter{section}{1}

\section{Statistical Learning}

\subsection*{\textit{Conceptual}}

\begin{enumerate}
  \item For each of parts (a) through (d), indicate whether we would generally
    expect the performance of a flexible statistical learning method to be
    better or worse than an inflexible method. Justify your answer.
    \begin{enumerate}
      \item When the sample size $n$ is extremely large and the
        number of predictors $p$ is small, a flexible method is
        \textbf{better}, since it is better able to account for the
        patterns present in the more abundant training data, with
        less risk of overfitting, since outliers are balanced out.
      \item When $p$ is extremely large and $n$ is small, a flexible
        method is \textbf{worse}, since it overfits to the
        hyper-specificity of the many parameters in the data, without
        a large sample size to support any inferred patterns. In
        these scenarios, flexible models are highly inconsistent and
        have high variance; that is, they depend strongly on the
        amount---or lack thereof---of training data.
      \item When the relationship between the predictors and response
        is highly non-linear, a flexible method is \textbf{better},
        since it is able to account for nuance, while inflexible
        methods remain rigid.
      \item When the variance of the error terms, i.e. $\sigma^2 =
        \text{Var}(\epsilon)$, is extremely high, a flexible method
        is \textbf{worse}, since it will be overly influenced by the
        noise from the training data and prone to overfitting, while
        an inflexible method is more stable and better at
        generalizing to unseen data.
    \end{enumerate}

  \item Explain whether each scenario is a classification or
    regression problem, and indicate whether we are most interested
    in inference or prediction. Finally, provide $n$ and $p$.
    \begin{enumerate}
      \item We collect a set of data on the top 500 firms in the US. For each
        firm we record profit, number of employees, industry and the
        CEO salary. We are interested in understanding which factors
        affect CEO salary.

        \begin{answer}
          This is a \textbf{regression} problem dealing with a
          quantitative \texttt{salary}. $n$ is 500 and $p$ is 3.
        \end{answer}
      \item We are considering launching a new product and wish to know
        whether it will be a \textit{success} or a \textit{failure}.
        We collect data on 20 similar products that were previously
        launched. For each
        product we have recorded whether it was a success or failure, price
        charged for the product, marketing budget, competition price,
        and ten other variables.

        \begin{answer}
          This is a \textbf{classification} problem with two
          categories as output,
          \texttt{success} and \texttt{failure}. $n$ is 20 and $p$ is 13.
        \end{answer}
      \item We are interested in predicting the \% change in the USD/Euro
        exchange rate in relation to the weekly changes in the world
        stock markets. Hence we collect weekly data for all of 2012. For
        each week we record the \% change in the USD/Euro, the \%
        change in the US market, the \% change in the British market,
        and the \% change in the German market.

        \begin{answer}
          This is a \textbf{regression} problem dealing with
          quantitative percent change. $n$ is $52$ (weeks in 2012),
          and $p$ is 3.
        \end{answer}
    \end{enumerate}

  \item We now revisit the bias-variance decomposition.
    \begin{enumerate}
      \item
        %   \begin{figure}[ht]
        %     \centering
        %     \incfig{ch2_3a_bias_variance}
        %     \caption{test}
        %     \label{fig:bias-variance}
        % \end{figure}
      \item
    \end{enumerate}

  \item You will now think of some real-life applications for
    statistical learning.
    \begin{enumerate}
      \item Describe three real-life applications in which classification might
        be useful. Describe the response, as well as the predictors. Is the
        goal of each application inference or prediction? Explain your
        answer.
        \begin{itemize}
          \item Collecting students' academic data to
            improve effectiveness of teaching strategies. Responses
            could include indicators like grade point average or
            standardized test scores, and predictors could include
            teacher-to-student ratio, class length, average minutes
            of each type of classroom activity (e.g. group work,
            lectures, labs) and amount of daily homework assigned.
            This is an \textbf{inference} problem, since it is
            important to interpret which of the predictors is most
            vital to student success.
        \end{itemize}

      \item Describe three real-life applications in which regression might
        be useful. Describe the response, as well as the predictors. Is the
        goal of each application inference or prediction? Explain your
        answer.
        \begin{itemize}
          \item Predicting weather. Predictors could include
            temperature, pressure, wind, precipitation, and humidity
            for a specific point in time. Responses could include any
            of those same predictor variables. This is a
            \textbf{prediction} application, since it is more
            important to get an accurate forecast than to understand
            what variables cause the forecast and their relationships.
        \end{itemize}
      \item Describe three real-life applications in which cluster analysis
        might be useful.
        \begin{itemize}
          \item Grouping professional sports players into archetypes
            by analyzing their in-game data. Useful for general
            managers who want to build more cohesive teams or fill a
            missing role.
        \end{itemize}
    \end{enumerate}

  \item What are the advantages and disadvantages of a very flexible (versus
    a less flexible) approach for regression or classification? Under what
    circumstances might a more flexible approach be preferred to a less
    flexible approach? When might a less flexible approach be preferred?
    \begin{enumerate}
      \item
      \item
      \item
    \end{enumerate}

  \item
    \begin{enumerate}
      \item
      \item
      \item
    \end{enumerate}

  \item
    \begin{enumerate}
      \item
      \item
      \item
    \end{enumerate}

\end{enumerate}

\subsection*{\textit{Applied}}

\begin{enumerate}[resume]
  \item
    \begin{enumerate}
      \item
      \item
      \item
    \end{enumerate}
\end{enumerate}

\end{document}
