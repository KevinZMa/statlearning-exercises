\documentclass{article}
\usepackage{amsmath}
\usepackage{hyperref}
\usepackage{lineno}
\usepackage[stretch=5,shrink=5]{microtype}
\usepackage{xcolor}

% Make line numbers smaller, lighter gray, and further to the left
\renewcommand\linenumberfont{\normalfont\tiny\color{gray}}
\setlength\linenumbersep{2em}

\linenumbers

\title{Exercises for ISLP\footnote{\textit{An Introduction to
    Statistical Learning with Applications in Python}:
\url{https://www.statlearning.com/}}}
\author{Kevin Ma}
\date{May 2025}

\begin{document}

\maketitle

\setcounter{section}{1}

\section{Statistical Learning}

\subsection*{\textit{Conceptual}}

\begin{enumerate}
  \item For each of parts (a) through (d), indicate whether we would generally
    expect the performance of a flexible statistical learning method to be
    better or worse than an inflexible method. Justify your answer.
    \begin{enumerate}
      \item When the sample size $n$ is extremely large and the
        number of predictors $p$ is small, a flexible method is
        \textbf{better}, since it is better able to account for the
        patterns present in the more abundant training data, with
        less risk of overfitting, since outliers are balanced out.
      \item When $p$ is extremely large and $n$ is small, a flexible
        method is \textbf{worse}, since it overfits to the
        hyper-specificity of the many parameters in the data, without
        a large sample size to support any inferred patterns. In
        these scenarios, flexible models are highly inconsistent and
        have high variance; that is, they depend strongly on the
        amount---or lack thereof---of training data.
      \item When the relationship between the predictors and response
        is highly non-linear, a flexible method is \textbf{better},
        since it is able to account for nuance, while inflexible
        methods remain rigid.
      \item When the variance of the error terms, i.e. $\sigma^2 =
        \text{Var}(\epsilon)$, is extremely high, a flexible method
        is \textbf{worse}, since it will be overly influenced by the
        noise from the training data and prone to overfitting, while
        an inflexible method is more stable and better at
        generalizing to unseen data.
    \end{enumerate}

  \item Explain whether each scenario is a classification or
    regression problem, and indicate whether we are most interested
    in inference or prediction. Finally, provide $n$ and $p$.
    \begin{enumerate}
      \item
      \item
      \item
    \end{enumerate}

  \item
    \begin{enumerate}
      \item
      \item
      \item
    \end{enumerate}

  \item
    \begin{enumerate}
      \item
      \item
      \item
    \end{enumerate}

  \item
    \begin{enumerate}
      \item
      \item
      \item
    \end{enumerate}

  \item
    \begin{enumerate}
      \item
      \item
      \item
    \end{enumerate}

  \item
    \begin{enumerate}
      \item
      \item
      \item
    \end{enumerate}

  \item
    \begin{enumerate}
      \item
      \item
      \item
    \end{enumerate}
\end{enumerate}

\end{document}
